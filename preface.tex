\section{序}
華夏、鷄林、扶桑之類、古以漢文爲國風之礎矣。
古欲習漢文、以字典爲寶。
字典雖有文字之解、未有詞彙之解。
各國、雖有其辭書者、以諺解漢。
他國之人視、畫中之餅也耳。
以漢解漢、四海蒼生之習漢文者、皆可以讀。
余願此書爲江湖諸賢所補、爲四海諸學所寶耳。
\section{範例}
此書之範例如左。
\par 第一條:「各項、取諸漢文正典、順之若康熙字典。」
\par 第二條:「新物之名、古道新作。」
\par 第三條:「詞之常同而綴之者、亦載之矣。」
\par 第四修:「變品之事、若從常道、不之載。」
