\chapter*{首}
夏、韓、和、越之風,基於漢文矣。
古欲習漢文也,以字典爲寶。
字典雖有字彙之釋而未有詞。
各國有其詞書,以諺釋文,
不識其國之諺,畫中之餅也耳。
進其所從文法,甚舊。
此書所以用新文法者,徒以舊文法之不調不和也。
文法之學始於希臘,雖與洋文相類,漢文不從,
近者雖借西洋文法,而斷鶴續鳧已\parencite{Ahn:2012}。
各項或取諸漢籍,或古道而新製。
各項順之康熙字典。空詞略之。
同源之詞,合項而載;異源之詞,分項而載。不載俗語。
以文釋文,詳述其用,弱通漢文,皆可以讀。
余願此書爲江湖諸賢所參,爲四海諸學所寶耳。
\chapter*{文法}
此書之所用以金琮鎬與甫立本爲基礎。
句有詞、節之分。
詞有指詞、謂詞、幹詞、成詞、束詞、助詞、冠詞、標詞、結詞之分\parencites[43-4]{Pulleyblank:2005}[2-35]{Kim:2019}。
指詞如『堯』『犬』『木』『秋』。
謂詞如『食』『去』『有』『如』。
幹詞非『使』,否則空。成詞如『嗚』『否』。
束詞如『之』,助詞如『如』,冠詞如『不』,標詞如『而』『也』,結詞如『及』『與』。
節有指節、謂節、幹節之別\parencite[15]{Kim:2019}。
指節具指句,具束詞,而具幹節。
謂節具謂句,具助詞,而具體節。
幹節具幹句,具冠詞,而具謂節。
成節具成句,具標詞,而具幹節。
詞有空者,字數零也\parencite[2]{Kim:2019}。
空詞亦有空體詞、空謂詞、空修詞、空助詞、空成詞、空束詞之分。
體詞所言之固也,可以空體詞代之;
謂詞所言之固也,可以空謂詞代之;
修詞所言之固也,可以空修詞代之;
助詞所言之固也,可以空助詞代之;
成詞所言之固也,可以空成詞代之;
束詞所言之固也,可以空束詞代之。
