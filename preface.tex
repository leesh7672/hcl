\section{序}
華夏、鷄林、扶桑之類、古以漢文爲國風之礎矣。
古欲習漢文、以字典爲寶。
字典雖有文字之解、未有詞彙之解。
各國、雖有其辭書者、以諺解漢。
他國之人視、畫中之餅也耳。
以漢解漢、四海蒼生之習漢文者、皆可以讀。
余願此書爲江湖諸賢所補、爲四海諸學所寶耳。
\section{範例}
此書之範例如左。
\par 第一條:「各項、取諸漢文正典、順之若康熙字典。」
\par 第二條:「新物之名、古道新作。」
\par 第三條:「詞之常同而綴之者、亦載之矣。」
\par 第四修:「變品之事、若從常道、不之載。」
\section{十二性}
\par 凡詞有十性之別如左。
\par 「馬」「山」「日」「堯」之類、謂之稱字。
\par 「自」「者」「彼」「我」之類、謂之指字。
\par 「去」「安」「眠」「爲」之類、謂之述字。
\par 「匹」「升」「位」「個」之類、謂之度字。
\par 「一」「甲」「獨」「單」之類、謂之數字。
\par 「其」「莫」「或」「非」之類、謂之冠字。
\par 「不」「以」「獨」「可」之類、謂之修字。
\par 「矣」「乎」「也」「哉」之類、謂之竟字。
\par 「而」「之」「於」「然」之類、謂之結字。
\section{八部}
\par 凡句有七部如左。
\par 「馬」之於「馬去。」、「家」之於「家和」、
「苦」之於「苦盡」、「今日」與「風波」之於「今日風波可甚」之類、謂之擧部。
\par 「突」之於「豬突」、「成」之於「萬事成」、「狗」之於「彼狗也」之類、謂之述部。
\par 「山」之於「上山」、「之」與「馬一匹」之於「與之馬一匹」、「遠方」之於「自遠方」、
「之」之於「未之有」之類、謂之補部。
\par 「此」之於「此魚」、「靑雲之」之於「靑雲之志」、「或」之於「或者」之類、謂之冠部。
\par 「自」之於「自在」、「甚」之於「風甚淸」之類、謂之修部。
\par 「天」之於「天乎、可高。」、「嗚乎」之於「嗚乎、可惜。」、「蓋」之於「蓋然。」之類、謂之發部。
\par 「也」之於「未之有也。」、「耳」之於「今少耳。」、謂之結部。
