\section{首}
華夏、雞林、扶桑之類,基於漢文矣。
古欲習漢文,以字典爲寶。
字典雖有字彙之解、未有詞。
各國雖有其詞書,以諺釋文。
不識其國之諺,畫中之餅也耳。
以文釋文,詳述其用,弱通漢文,皆可以讀。
余願此書爲江湖諸賢所參,爲四海諸學所寶耳。
\section{範}
此書之範如左。
\par 第一條:「各詞,或取諸漢籍,或古道而新製。」
\par 第二條:「各項,順之康熙字典。」
\par 第三條:「無雅無俗,以文解之。」
\par 第四條:「變品之詞,如從常道,不之分。」
\par 第五條:「同源同字同音之詞,同項而載,否則分項而載。」
\section{文法}
此書所以用新文法者,徒以舊文法之不調不和也。
文法之學始於希臘,雖與洋文相類,漢文不從。
近來華夏之學雖借拉丁文法,而斷鶴續鳧也已。
厥後東西諸學提新文法,例如甫立本、金琮鎬。
此書之所用以金琮鎬與甫立本爲基礎,而考易習,有所從舊文法。
詞有體詞、謂詞、狀詞、助詞、束詞、結詞之差。
節有體詞節、謂詞節、助詞節、結詞節之別。
體詞節有定語、體詞,助詞節有助詞、體語,謂詞節有題語、主語、狀語、謂詞、賓語、補語。結詞節有謂語、結詞。
定語成以謂詞之節,題語成以結詞之節,主語與賓語以體詞之節,狀語以謂詞之節,補語以助詞之節,體語以體詞之節,謂語以謂詞之節。
諸節萬詞,空詞可代之。題語與狀語之數可變。
