\chapter*{首}
夏、韓、和、越之風,基於漢文矣。
古欲習漢文也,以字典爲寶。
字典雖有字彙之釋而未有詞。
各國有其詞書,以諺釋文,
不識其國之諺,畫中之餅也耳。
進其所從文法,甚舊。
此書所以用新文法者,徒以舊文法之不調不和也。
文法之學始於希臘,雖與洋文相類,漢文不從,
近者雖借西洋文法,而斷鶴續鳧已\parencite{Ahn:2012}。
各項或取諸漢籍,或從古道而新製。
各項順之康熙字典。空詞略之。
同源之詞,合項而載;異源之詞,分項而載。不載俗語。
以文釋文,詳述其用,弱通漢文,皆可以讀。
余願此書爲江湖諸賢所參,爲四海諸學所寶耳。
