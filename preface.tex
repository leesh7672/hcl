\section{首}
華夏、雞林、扶桑之類,基於古經漢文矣。
古欲習經文,以字典爲寶。
字典雖有字彙之解、未有詞。
各國,雖有其詞書,以諺釋文。
不識其國之諺,畫中之餅也耳。
以文釋文,進釋華夏內外之諺,詳述其用,弱通漢文,皆可以讀。
余願此書爲江湖諸賢所參,爲四海諸學所寶耳。
\section{範}
此書之範如左。
\par 第一條:「各詞,或取諸漢籍,或古道而新製。」
\par 第一條:「各項,順之康熙字典。」
\par 第三條:「無雅無俗,以文解之。」
\par 第四條:「變品之詞,如從常道,不之載。」
\par 第五條:「同源同字同音之詞,同項而載,否則分項而載。」
\section{文法}
此書所以用新文法者,徒以舊文法之不調不和也。
文法之學始於希臘,雖與洋文相類,漢文不從。
近來華夏諸學,雖借羅甸文法,斷鶴之續鳧已。
厥後東西諸學提新文法,例如甫立本、金琮鎬。
此書之所用,以金琮鎬爲架構,以微言宗爲基礎。
