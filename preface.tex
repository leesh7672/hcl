\chapter*{首}
夏、韓、和、越之風,基於漢文矣。
古欲習漢文也,以字典爲寶。
字典雖有字彙之釋而未有詞。
各國雖有其詞書,以諺釋文,
不識其國之諺,畫中之餅也耳。
進其所從文法,甚舊。
此書所以用新文法者,徒以舊文法之不調不和也。
文法之學始於希臘,雖與洋文相類,漢文不從,
近者雖借西洋文法,而斷鶴續鳧已\parencite{Ahn:2012}。
以文釋文,進釋中華內外之諺,詳述其用,弱通漢文,皆可以讀。
余願此書爲江湖諸賢所參,爲四海諸學所寶耳。
各項或取諸漢籍,或古道而新製。
各項順之康熙字典。空詞略之。
同源之詞,合項而載;異源之詞,分項而載。
詞無雅俗皆載。
\chapter*{文法}
此書之所用以金琮鎬與甫立本爲基礎,而考易習。
詞有體詞、謂詞、杆詞、助詞、補詞、標詞、刻詞之差\parencite[][2-35]{Kim:2019}。
節有體節、謂節、杆節、助節、補節之別\parencite[][15]{Kim:2019}。
體節有體詞與結節,
助節有助詞與體節。
謂節或有謂詞與體節,否則有謂節與助節。
杆節或有杆詞與謂節,否則有謂節與補節。
補節有補詞與謂節。
俗語有核詞於謂詞之後,雅語無之。
詞有空者,字數零也\parencite[][2]{Kim:2019}。空詞亦有體詞、謂詞、冠詞、助詞、結詞之差。
體詞、謂詞、助詞、結詞所言之確然也,空詞可代之。
謂詞所言之實也,冠詞槪空。
