\section{首}
華夏、雞林、扶桑之類,基於漢文矣。
古欲習漢文,以字典爲寶。
字典雖有字彙之解、未有詞。
各國雖有其詞書,以諺釋文。
不識其國之諺,畫中之餅也耳。
以文釋文,詳述其用,弱通漢文,皆可以讀。
余願此書爲江湖諸賢所參,爲四海諸學所寶耳。
\section{範}
此書之範如左。
\par 第一條:「各詞,或取諸漢籍,或古道而新製。」
\par 第二條:「各項,順之康熙字典。」
\par 第三條:「無雅無俗,以文解之。」
\par 第四條:「變品之詞,如從常道,不之分。」
\par 第五條:「同源同字同音之詞,同項而載,否則分項而載。」
\section{文法}
此書所以用新文法者,徒以舊文法之不調不和也。
文法之學始於希臘,雖與洋文相類,漢文不從。
近來華夏之學雖借拉丁文法,而斷鶴續鳧也已。
厥後東西諸學提新文法,例如甫立本、金琮鎬。
此書之所用以金琮鎬與甫立本爲基礎,而考易習,有所從舊文法。
漢文之詞有體詞、謂詞、冠詞、標詞之差。
漢文之句有主詞、謂詞、賓詞、補詞,而分行動態、變動態、使動態、靜態。
句片成以詞及節,否則以空詞。句有甲式、乙式、丙式之分。
甲式之句有主詞、謂詞、賓詞,乙式之句有主詞、謂詞、補詞,丙式之句有主詞、謂詞、賓詞、補詞。
句片之間有序,上有主詞;下有謂詞。賓詞與補詞槪在謂詞之下,時而上謂詞。其所以變序者,隋詞之爲何,亦隨時風。
