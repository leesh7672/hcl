\documentclass[a5paper,landscape,10pt]{memoir}
\usepackage{fontspec}
\usepackage[top=10mm,left=10mm,bottom=10mm,right=10mm]{geometry}
\usepackage{multicol}
\usepackage{setspace}
\usepackage[nospace]{xeCJK}
\usepackage{CJKnumb}
\usepackage{nameref}
\usepackage{makeidx}
\pagestyle{empty}
\newcommand*\CJKmovesymbol[1]{\raise.35em\hbox{#1}}
\newcommand*\CJKmove{\punctstyle{plain}
		       \let\CJKsymbol\CJKmovesymbol
			   \let\CJKpunctsymbol\CJKsymbol}
\setmainfont[RawFeature=vertical,Vertical=RotatedGlyphs]{Noto Serif CJK KR}
\setCJKmainfont[RawFeature=vertical,Vertical=RotatedGlyphs]{Noto Serif CJK KR}
\newcommand{\explain}[2]{\\{{#1}}\\{$\bullet$}{#2}}
\newcommand{\go}[2]{\\{$\bullet$}請考\textbf{#2} 矣。 }
\newcommand{\entry}[4]{\par{\begin{minipage}{0.2\textwidth}{{\vspace{2mm}\hspace{4mm}{\bfseries{#1}}{\textsuperscript{\CJKnumber{#2}}}{#3}}}\end{minipage}}\\}
\newcommand{\also}[1]{\\或書之\textbf{#1}。}
\newcommand{\samp}[2]{{#1}云、「{#2}」}
\newcommand{\syn}[2]{與\textbf{#1}\textsuperscript{\CJKnumber{#2}}相通。}
\newcommand{\ant}[2]{與\textbf{#1}\textsuperscript{\CJKnumber{#2}}相對。}
\begin{document}
\frontmatter
\hspace{0pt}
\vfill
{\vspace{14pt}\fontsize{48pt}{0pt}{漢辭林}}
\vfill
\hspace{18pt}
\newpage
\linespread{1.5}
\section{序}
華夏、雞林、扶桑之類、古以漢文爲國風之礎矣。
古欲習漢文、以字典爲寶。
字典雖有文字之解、未有詞彙之解。
各國、雖有其辭書者、以諺解漢。
他國之人視、畫中之餅也耳。
以漢解漢、四海蒼生之習漢文者、皆可以讀。
余願此書爲江湖諸賢所補、爲四海諸學所寶耳。
\section{範例}
此書之範例如左。
\par 第一條:「各項、取諸漢文古典、順之若康熙字典。」
\par 第二條:「諺字與邦字、載而明之。」
\par 第三條:「詞之常同而綴之者、亦載之矣。」
\par 第四修:「變品之事、若從常道、不之載。」

\mainmatter
\linespread{1.5}
\begin{multicols}{4}
\input{hcl-entries.tex}
\end{multicols}
\end{document}
